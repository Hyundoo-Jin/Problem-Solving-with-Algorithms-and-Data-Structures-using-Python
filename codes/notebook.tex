
% Default to the notebook output style

    


% Inherit from the specified cell style.




    
\documentclass[11pt]{article}

    
    
    \usepackage[T1]{fontenc}
    % Nicer default font (+ math font) than Computer Modern for most use cases
    \usepackage{mathpazo}

    % Basic figure setup, for now with no caption control since it's done
    % automatically by Pandoc (which extracts ![](path) syntax from Markdown).
    \usepackage{graphicx}
    % We will generate all images so they have a width \maxwidth. This means
    % that they will get their normal width if they fit onto the page, but
    % are scaled down if they would overflow the margins.
    \makeatletter
    \def\maxwidth{\ifdim\Gin@nat@width>\linewidth\linewidth
    \else\Gin@nat@width\fi}
    \makeatother
    \let\Oldincludegraphics\includegraphics
    % Set max figure width to be 80% of text width, for now hardcoded.
    \renewcommand{\includegraphics}[1]{\Oldincludegraphics[width=.8\maxwidth]{#1}}
    % Ensure that by default, figures have no caption (until we provide a
    % proper Figure object with a Caption API and a way to capture that
    % in the conversion process - todo).
    \usepackage{caption}
    \DeclareCaptionLabelFormat{nolabel}{}
    \captionsetup{labelformat=nolabel}

    \usepackage{adjustbox} % Used to constrain images to a maximum size 
    \usepackage{xcolor} % Allow colors to be defined
    \usepackage{enumerate} % Needed for markdown enumerations to work
    \usepackage{geometry} % Used to adjust the document margins
    \usepackage{amsmath} % Equations
    \usepackage{amssymb} % Equations
    \usepackage{textcomp} % defines textquotesingle
    % Hack from http://tex.stackexchange.com/a/47451/13684:
    \AtBeginDocument{%
        \def\PYZsq{\textquotesingle}% Upright quotes in Pygmentized code
    }
    \usepackage{upquote} % Upright quotes for verbatim code
    \usepackage{eurosym} % defines \euro
    \usepackage[mathletters]{ucs} % Extended unicode (utf-8) support
    \usepackage[utf8x]{inputenc} % Allow utf-8 characters in the tex document
    \usepackage{fancyvrb} % verbatim replacement that allows latex
    \usepackage{grffile} % extends the file name processing of package graphics 
                         % to support a larger range 
    % The hyperref package gives us a pdf with properly built
    % internal navigation ('pdf bookmarks' for the table of contents,
    % internal cross-reference links, web links for URLs, etc.)
    \usepackage{hyperref}
    \usepackage{longtable} % longtable support required by pandoc >1.10
    \usepackage{booktabs}  % table support for pandoc > 1.12.2
    \usepackage[inline]{enumitem} % IRkernel/repr support (it uses the enumerate* environment)
    \usepackage[normalem]{ulem} % ulem is needed to support strikethroughs (\sout)
                                % normalem makes italics be italics, not underlines
    

    
    
    % Colors for the hyperref package
    \definecolor{urlcolor}{rgb}{0,.145,.698}
    \definecolor{linkcolor}{rgb}{.71,0.21,0.01}
    \definecolor{citecolor}{rgb}{.12,.54,.11}

    % ANSI colors
    \definecolor{ansi-black}{HTML}{3E424D}
    \definecolor{ansi-black-intense}{HTML}{282C36}
    \definecolor{ansi-red}{HTML}{E75C58}
    \definecolor{ansi-red-intense}{HTML}{B22B31}
    \definecolor{ansi-green}{HTML}{00A250}
    \definecolor{ansi-green-intense}{HTML}{007427}
    \definecolor{ansi-yellow}{HTML}{DDB62B}
    \definecolor{ansi-yellow-intense}{HTML}{B27D12}
    \definecolor{ansi-blue}{HTML}{208FFB}
    \definecolor{ansi-blue-intense}{HTML}{0065CA}
    \definecolor{ansi-magenta}{HTML}{D160C4}
    \definecolor{ansi-magenta-intense}{HTML}{A03196}
    \definecolor{ansi-cyan}{HTML}{60C6C8}
    \definecolor{ansi-cyan-intense}{HTML}{258F8F}
    \definecolor{ansi-white}{HTML}{C5C1B4}
    \definecolor{ansi-white-intense}{HTML}{A1A6B2}

    % commands and environments needed by pandoc snippets
    % extracted from the output of `pandoc -s`
    \providecommand{\tightlist}{%
      \setlength{\itemsep}{0pt}\setlength{\parskip}{0pt}}
    \DefineVerbatimEnvironment{Highlighting}{Verbatim}{commandchars=\\\{\}}
    % Add ',fontsize=\small' for more characters per line
    \newenvironment{Shaded}{}{}
    \newcommand{\KeywordTok}[1]{\textcolor[rgb]{0.00,0.44,0.13}{\textbf{{#1}}}}
    \newcommand{\DataTypeTok}[1]{\textcolor[rgb]{0.56,0.13,0.00}{{#1}}}
    \newcommand{\DecValTok}[1]{\textcolor[rgb]{0.25,0.63,0.44}{{#1}}}
    \newcommand{\BaseNTok}[1]{\textcolor[rgb]{0.25,0.63,0.44}{{#1}}}
    \newcommand{\FloatTok}[1]{\textcolor[rgb]{0.25,0.63,0.44}{{#1}}}
    \newcommand{\CharTok}[1]{\textcolor[rgb]{0.25,0.44,0.63}{{#1}}}
    \newcommand{\StringTok}[1]{\textcolor[rgb]{0.25,0.44,0.63}{{#1}}}
    \newcommand{\CommentTok}[1]{\textcolor[rgb]{0.38,0.63,0.69}{\textit{{#1}}}}
    \newcommand{\OtherTok}[1]{\textcolor[rgb]{0.00,0.44,0.13}{{#1}}}
    \newcommand{\AlertTok}[1]{\textcolor[rgb]{1.00,0.00,0.00}{\textbf{{#1}}}}
    \newcommand{\FunctionTok}[1]{\textcolor[rgb]{0.02,0.16,0.49}{{#1}}}
    \newcommand{\RegionMarkerTok}[1]{{#1}}
    \newcommand{\ErrorTok}[1]{\textcolor[rgb]{1.00,0.00,0.00}{\textbf{{#1}}}}
    \newcommand{\NormalTok}[1]{{#1}}
    
    % Additional commands for more recent versions of Pandoc
    \newcommand{\ConstantTok}[1]{\textcolor[rgb]{0.53,0.00,0.00}{{#1}}}
    \newcommand{\SpecialCharTok}[1]{\textcolor[rgb]{0.25,0.44,0.63}{{#1}}}
    \newcommand{\VerbatimStringTok}[1]{\textcolor[rgb]{0.25,0.44,0.63}{{#1}}}
    \newcommand{\SpecialStringTok}[1]{\textcolor[rgb]{0.73,0.40,0.53}{{#1}}}
    \newcommand{\ImportTok}[1]{{#1}}
    \newcommand{\DocumentationTok}[1]{\textcolor[rgb]{0.73,0.13,0.13}{\textit{{#1}}}}
    \newcommand{\AnnotationTok}[1]{\textcolor[rgb]{0.38,0.63,0.69}{\textbf{\textit{{#1}}}}}
    \newcommand{\CommentVarTok}[1]{\textcolor[rgb]{0.38,0.63,0.69}{\textbf{\textit{{#1}}}}}
    \newcommand{\VariableTok}[1]{\textcolor[rgb]{0.10,0.09,0.49}{{#1}}}
    \newcommand{\ControlFlowTok}[1]{\textcolor[rgb]{0.00,0.44,0.13}{\textbf{{#1}}}}
    \newcommand{\OperatorTok}[1]{\textcolor[rgb]{0.40,0.40,0.40}{{#1}}}
    \newcommand{\BuiltInTok}[1]{{#1}}
    \newcommand{\ExtensionTok}[1]{{#1}}
    \newcommand{\PreprocessorTok}[1]{\textcolor[rgb]{0.74,0.48,0.00}{{#1}}}
    \newcommand{\AttributeTok}[1]{\textcolor[rgb]{0.49,0.56,0.16}{{#1}}}
    \newcommand{\InformationTok}[1]{\textcolor[rgb]{0.38,0.63,0.69}{\textbf{\textit{{#1}}}}}
    \newcommand{\WarningTok}[1]{\textcolor[rgb]{0.38,0.63,0.69}{\textbf{\textit{{#1}}}}}
    
    
    % Define a nice break command that doesn't care if a line doesn't already
    % exist.
    \def\br{\hspace*{\fill} \\* }
    % Math Jax compatability definitions
    \def\gt{>}
    \def\lt{<}
    % Document parameters
    \title{Chapter 3}
    
    
    

    % Pygments definitions
    
\makeatletter
\def\PY@reset{\let\PY@it=\relax \let\PY@bf=\relax%
    \let\PY@ul=\relax \let\PY@tc=\relax%
    \let\PY@bc=\relax \let\PY@ff=\relax}
\def\PY@tok#1{\csname PY@tok@#1\endcsname}
\def\PY@toks#1+{\ifx\relax#1\empty\else%
    \PY@tok{#1}\expandafter\PY@toks\fi}
\def\PY@do#1{\PY@bc{\PY@tc{\PY@ul{%
    \PY@it{\PY@bf{\PY@ff{#1}}}}}}}
\def\PY#1#2{\PY@reset\PY@toks#1+\relax+\PY@do{#2}}

\expandafter\def\csname PY@tok@w\endcsname{\def\PY@tc##1{\textcolor[rgb]{0.73,0.73,0.73}{##1}}}
\expandafter\def\csname PY@tok@c\endcsname{\let\PY@it=\textit\def\PY@tc##1{\textcolor[rgb]{0.25,0.50,0.50}{##1}}}
\expandafter\def\csname PY@tok@cp\endcsname{\def\PY@tc##1{\textcolor[rgb]{0.74,0.48,0.00}{##1}}}
\expandafter\def\csname PY@tok@k\endcsname{\let\PY@bf=\textbf\def\PY@tc##1{\textcolor[rgb]{0.00,0.50,0.00}{##1}}}
\expandafter\def\csname PY@tok@kp\endcsname{\def\PY@tc##1{\textcolor[rgb]{0.00,0.50,0.00}{##1}}}
\expandafter\def\csname PY@tok@kt\endcsname{\def\PY@tc##1{\textcolor[rgb]{0.69,0.00,0.25}{##1}}}
\expandafter\def\csname PY@tok@o\endcsname{\def\PY@tc##1{\textcolor[rgb]{0.40,0.40,0.40}{##1}}}
\expandafter\def\csname PY@tok@ow\endcsname{\let\PY@bf=\textbf\def\PY@tc##1{\textcolor[rgb]{0.67,0.13,1.00}{##1}}}
\expandafter\def\csname PY@tok@nb\endcsname{\def\PY@tc##1{\textcolor[rgb]{0.00,0.50,0.00}{##1}}}
\expandafter\def\csname PY@tok@nf\endcsname{\def\PY@tc##1{\textcolor[rgb]{0.00,0.00,1.00}{##1}}}
\expandafter\def\csname PY@tok@nc\endcsname{\let\PY@bf=\textbf\def\PY@tc##1{\textcolor[rgb]{0.00,0.00,1.00}{##1}}}
\expandafter\def\csname PY@tok@nn\endcsname{\let\PY@bf=\textbf\def\PY@tc##1{\textcolor[rgb]{0.00,0.00,1.00}{##1}}}
\expandafter\def\csname PY@tok@ne\endcsname{\let\PY@bf=\textbf\def\PY@tc##1{\textcolor[rgb]{0.82,0.25,0.23}{##1}}}
\expandafter\def\csname PY@tok@nv\endcsname{\def\PY@tc##1{\textcolor[rgb]{0.10,0.09,0.49}{##1}}}
\expandafter\def\csname PY@tok@no\endcsname{\def\PY@tc##1{\textcolor[rgb]{0.53,0.00,0.00}{##1}}}
\expandafter\def\csname PY@tok@nl\endcsname{\def\PY@tc##1{\textcolor[rgb]{0.63,0.63,0.00}{##1}}}
\expandafter\def\csname PY@tok@ni\endcsname{\let\PY@bf=\textbf\def\PY@tc##1{\textcolor[rgb]{0.60,0.60,0.60}{##1}}}
\expandafter\def\csname PY@tok@na\endcsname{\def\PY@tc##1{\textcolor[rgb]{0.49,0.56,0.16}{##1}}}
\expandafter\def\csname PY@tok@nt\endcsname{\let\PY@bf=\textbf\def\PY@tc##1{\textcolor[rgb]{0.00,0.50,0.00}{##1}}}
\expandafter\def\csname PY@tok@nd\endcsname{\def\PY@tc##1{\textcolor[rgb]{0.67,0.13,1.00}{##1}}}
\expandafter\def\csname PY@tok@s\endcsname{\def\PY@tc##1{\textcolor[rgb]{0.73,0.13,0.13}{##1}}}
\expandafter\def\csname PY@tok@sd\endcsname{\let\PY@it=\textit\def\PY@tc##1{\textcolor[rgb]{0.73,0.13,0.13}{##1}}}
\expandafter\def\csname PY@tok@si\endcsname{\let\PY@bf=\textbf\def\PY@tc##1{\textcolor[rgb]{0.73,0.40,0.53}{##1}}}
\expandafter\def\csname PY@tok@se\endcsname{\let\PY@bf=\textbf\def\PY@tc##1{\textcolor[rgb]{0.73,0.40,0.13}{##1}}}
\expandafter\def\csname PY@tok@sr\endcsname{\def\PY@tc##1{\textcolor[rgb]{0.73,0.40,0.53}{##1}}}
\expandafter\def\csname PY@tok@ss\endcsname{\def\PY@tc##1{\textcolor[rgb]{0.10,0.09,0.49}{##1}}}
\expandafter\def\csname PY@tok@sx\endcsname{\def\PY@tc##1{\textcolor[rgb]{0.00,0.50,0.00}{##1}}}
\expandafter\def\csname PY@tok@m\endcsname{\def\PY@tc##1{\textcolor[rgb]{0.40,0.40,0.40}{##1}}}
\expandafter\def\csname PY@tok@gh\endcsname{\let\PY@bf=\textbf\def\PY@tc##1{\textcolor[rgb]{0.00,0.00,0.50}{##1}}}
\expandafter\def\csname PY@tok@gu\endcsname{\let\PY@bf=\textbf\def\PY@tc##1{\textcolor[rgb]{0.50,0.00,0.50}{##1}}}
\expandafter\def\csname PY@tok@gd\endcsname{\def\PY@tc##1{\textcolor[rgb]{0.63,0.00,0.00}{##1}}}
\expandafter\def\csname PY@tok@gi\endcsname{\def\PY@tc##1{\textcolor[rgb]{0.00,0.63,0.00}{##1}}}
\expandafter\def\csname PY@tok@gr\endcsname{\def\PY@tc##1{\textcolor[rgb]{1.00,0.00,0.00}{##1}}}
\expandafter\def\csname PY@tok@ge\endcsname{\let\PY@it=\textit}
\expandafter\def\csname PY@tok@gs\endcsname{\let\PY@bf=\textbf}
\expandafter\def\csname PY@tok@gp\endcsname{\let\PY@bf=\textbf\def\PY@tc##1{\textcolor[rgb]{0.00,0.00,0.50}{##1}}}
\expandafter\def\csname PY@tok@go\endcsname{\def\PY@tc##1{\textcolor[rgb]{0.53,0.53,0.53}{##1}}}
\expandafter\def\csname PY@tok@gt\endcsname{\def\PY@tc##1{\textcolor[rgb]{0.00,0.27,0.87}{##1}}}
\expandafter\def\csname PY@tok@err\endcsname{\def\PY@bc##1{\setlength{\fboxsep}{0pt}\fcolorbox[rgb]{1.00,0.00,0.00}{1,1,1}{\strut ##1}}}
\expandafter\def\csname PY@tok@kc\endcsname{\let\PY@bf=\textbf\def\PY@tc##1{\textcolor[rgb]{0.00,0.50,0.00}{##1}}}
\expandafter\def\csname PY@tok@kd\endcsname{\let\PY@bf=\textbf\def\PY@tc##1{\textcolor[rgb]{0.00,0.50,0.00}{##1}}}
\expandafter\def\csname PY@tok@kn\endcsname{\let\PY@bf=\textbf\def\PY@tc##1{\textcolor[rgb]{0.00,0.50,0.00}{##1}}}
\expandafter\def\csname PY@tok@kr\endcsname{\let\PY@bf=\textbf\def\PY@tc##1{\textcolor[rgb]{0.00,0.50,0.00}{##1}}}
\expandafter\def\csname PY@tok@bp\endcsname{\def\PY@tc##1{\textcolor[rgb]{0.00,0.50,0.00}{##1}}}
\expandafter\def\csname PY@tok@fm\endcsname{\def\PY@tc##1{\textcolor[rgb]{0.00,0.00,1.00}{##1}}}
\expandafter\def\csname PY@tok@vc\endcsname{\def\PY@tc##1{\textcolor[rgb]{0.10,0.09,0.49}{##1}}}
\expandafter\def\csname PY@tok@vg\endcsname{\def\PY@tc##1{\textcolor[rgb]{0.10,0.09,0.49}{##1}}}
\expandafter\def\csname PY@tok@vi\endcsname{\def\PY@tc##1{\textcolor[rgb]{0.10,0.09,0.49}{##1}}}
\expandafter\def\csname PY@tok@vm\endcsname{\def\PY@tc##1{\textcolor[rgb]{0.10,0.09,0.49}{##1}}}
\expandafter\def\csname PY@tok@sa\endcsname{\def\PY@tc##1{\textcolor[rgb]{0.73,0.13,0.13}{##1}}}
\expandafter\def\csname PY@tok@sb\endcsname{\def\PY@tc##1{\textcolor[rgb]{0.73,0.13,0.13}{##1}}}
\expandafter\def\csname PY@tok@sc\endcsname{\def\PY@tc##1{\textcolor[rgb]{0.73,0.13,0.13}{##1}}}
\expandafter\def\csname PY@tok@dl\endcsname{\def\PY@tc##1{\textcolor[rgb]{0.73,0.13,0.13}{##1}}}
\expandafter\def\csname PY@tok@s2\endcsname{\def\PY@tc##1{\textcolor[rgb]{0.73,0.13,0.13}{##1}}}
\expandafter\def\csname PY@tok@sh\endcsname{\def\PY@tc##1{\textcolor[rgb]{0.73,0.13,0.13}{##1}}}
\expandafter\def\csname PY@tok@s1\endcsname{\def\PY@tc##1{\textcolor[rgb]{0.73,0.13,0.13}{##1}}}
\expandafter\def\csname PY@tok@mb\endcsname{\def\PY@tc##1{\textcolor[rgb]{0.40,0.40,0.40}{##1}}}
\expandafter\def\csname PY@tok@mf\endcsname{\def\PY@tc##1{\textcolor[rgb]{0.40,0.40,0.40}{##1}}}
\expandafter\def\csname PY@tok@mh\endcsname{\def\PY@tc##1{\textcolor[rgb]{0.40,0.40,0.40}{##1}}}
\expandafter\def\csname PY@tok@mi\endcsname{\def\PY@tc##1{\textcolor[rgb]{0.40,0.40,0.40}{##1}}}
\expandafter\def\csname PY@tok@il\endcsname{\def\PY@tc##1{\textcolor[rgb]{0.40,0.40,0.40}{##1}}}
\expandafter\def\csname PY@tok@mo\endcsname{\def\PY@tc##1{\textcolor[rgb]{0.40,0.40,0.40}{##1}}}
\expandafter\def\csname PY@tok@ch\endcsname{\let\PY@it=\textit\def\PY@tc##1{\textcolor[rgb]{0.25,0.50,0.50}{##1}}}
\expandafter\def\csname PY@tok@cm\endcsname{\let\PY@it=\textit\def\PY@tc##1{\textcolor[rgb]{0.25,0.50,0.50}{##1}}}
\expandafter\def\csname PY@tok@cpf\endcsname{\let\PY@it=\textit\def\PY@tc##1{\textcolor[rgb]{0.25,0.50,0.50}{##1}}}
\expandafter\def\csname PY@tok@c1\endcsname{\let\PY@it=\textit\def\PY@tc##1{\textcolor[rgb]{0.25,0.50,0.50}{##1}}}
\expandafter\def\csname PY@tok@cs\endcsname{\let\PY@it=\textit\def\PY@tc##1{\textcolor[rgb]{0.25,0.50,0.50}{##1}}}

\def\PYZbs{\char`\\}
\def\PYZus{\char`\_}
\def\PYZob{\char`\{}
\def\PYZcb{\char`\}}
\def\PYZca{\char`\^}
\def\PYZam{\char`\&}
\def\PYZlt{\char`\<}
\def\PYZgt{\char`\>}
\def\PYZsh{\char`\#}
\def\PYZpc{\char`\%}
\def\PYZdl{\char`\$}
\def\PYZhy{\char`\-}
\def\PYZsq{\char`\'}
\def\PYZdq{\char`\"}
\def\PYZti{\char`\~}
% for compatibility with earlier versions
\def\PYZat{@}
\def\PYZlb{[}
\def\PYZrb{]}
\makeatother


    % Exact colors from NB
    \definecolor{incolor}{rgb}{0.0, 0.0, 0.5}
    \definecolor{outcolor}{rgb}{0.545, 0.0, 0.0}



    
    % Prevent overflowing lines due to hard-to-break entities
    \sloppy 
    % Setup hyperref package
    \hypersetup{
      breaklinks=true,  % so long urls are correctly broken across lines
      colorlinks=true,
      urlcolor=urlcolor,
      linkcolor=linkcolor,
      citecolor=citecolor,
      }
    % Slightly bigger margins than the latex defaults
    
    \geometry{verbose,tmargin=1in,bmargin=1in,lmargin=1in,rmargin=1in}
    
    

    \begin{document}
    
    
    \maketitle
    
    

    
    \subsubsection{3.2 Linear Structures}\label{linear-structures}

    Stacks, Queues, Deques, Lists의 네 가지 Data Structure에 대해 공부할
것이다. 이 데이터 구조를 Linear Structure(선형구조)라고 하는데, 이는
데이터가 다른 데이터들과 위치상의 상관관계(즉, 순서)를 가지고 있는 것을
말한다.

Linear Structure는 두 가지의 끝 지점을 가지고 있다. 왼쪽 끝과 오른쪽 끝,
혹은 앞과 뒤로 나누거나, top과 base로도 구분할 수 있다. 위의 나열된 네
가지 데이터구조는 데이터를 추가하고 제거하는 방식의 차이에 따라 구분할
수 있다.

    \subsubsection{3.3 What is a Stack?}\label{what-is-a-stack}

    Stack이란 item의 추가와 삭제가 항상 동일한 end에서 발생하는
자료구조이다.(데이터 추가와 삭제 모두 하나의 end에서만 발생한다.) item을
삭제할 때 가장 최근에 추가했던 데이터가 삭제되는 \textbf{LIFO(Last-in,
first-out)} 방식의 자료구조이다.

이를 이해하기 위한 가장 쉬운 예시 중 하나는 책상에 책 쌓아올리기다.
책상에 책을 모두 쌓아올린 뒤, 다시 책을 꺼낼 때엔 어떻게 하는가? 가장
위에 있는 책(가장 최근에 쌓아올린 책)부터 책을 꺼낼 것이다. 또, 인터넷
브라우저(크롬, IE 등)로 웹서핑을 할 때를 생각해보자. 뒤로가기 버튼은
항상 바로 이전의 웹페이지(가장 최근에 탐색했던 페이지)로 이동하게 한다.
계속 뒤로가기 버튼을 누르다 보면 언젠가 홈 웹페이지(base end의 페이지)로
되돌아갈 것이다.

    \subsubsection{3.4 The Stack Abstract Data
Type}\label{the-stack-abstract-data-type}

    stack abstract 데이터타입은 다음과같은 명령어로 구성되어 있다.

    \begin{itemize}
\tightlist
\item
  \texttt{Stack()} : 비어있는 새로운 스택을 반환한다. parameter는
  필요하지 않다.
\item
  \texttt{push(item)} : 새로운 item을 stack의 top에 넣는다. item param가
  필요하고, return은 없다.
\item
  \texttt{pop()} : stack의 top에 있는 item을 반환한다. param는 필요하지
  않고 stack에서 반환된 item이 삭제된다.\\
\item
  \texttt{peek()} : stack의 top에 있는 item을 반환하지만 stack은
  수정되지 않는다.\\
\item
  \texttt{isEmpty()} : stack이 비어있는지를 반환한다. boolean값으로
  반환한다.\\
\item
  \texttt{size()} : stack에 존재하는 item 개수를 반환한다. integer값을
  반환한다.
\end{itemize}

    다음은 stack 자료구조를 생성하고, 여러 method 실행결과와 stack 상태를
나타낸 것이다.

    \begin{Verbatim}[commandchars=\\\{\}]
{\color{incolor}In [{\color{incolor}1}]:} \PY{k+kn}{from} \PY{n+nn}{structures}\PY{n+nn}{.}\PY{n+nn}{Stack} \PY{k}{import} \PY{n}{Stack}
\end{Verbatim}


    \begin{Verbatim}[commandchars=\\\{\}]
{\color{incolor}In [{\color{incolor}2}]:} \PY{n}{s} \PY{o}{=} \PY{n}{Stack}\PY{p}{(}\PY{p}{)}
\end{Verbatim}


    \begin{Verbatim}[commandchars=\\\{\}]
{\color{incolor}In [{\color{incolor}3}]:} \PY{n+nb}{print}\PY{p}{(}\PY{n}{s}\PY{o}{.}\PY{n}{isEmpty}\PY{p}{(}\PY{p}{)}\PY{p}{,} \PY{n}{s}\PY{o}{.}\PY{n}{items}\PY{p}{)}
\end{Verbatim}


    \begin{Verbatim}[commandchars=\\\{\}]
True []

    \end{Verbatim}

    \begin{Verbatim}[commandchars=\\\{\}]
{\color{incolor}In [{\color{incolor}4}]:} \PY{n+nb}{print}\PY{p}{(}\PY{n}{s}\PY{o}{.}\PY{n}{push}\PY{p}{(}\PY{l+m+mi}{4}\PY{p}{)}\PY{p}{,} \PY{n}{s}\PY{o}{.}\PY{n}{items}\PY{p}{)}
\end{Verbatim}


    \begin{Verbatim}[commandchars=\\\{\}]
None [4]

    \end{Verbatim}

    \begin{Verbatim}[commandchars=\\\{\}]
{\color{incolor}In [{\color{incolor}5}]:} \PY{n+nb}{print}\PY{p}{(}\PY{n}{s}\PY{o}{.}\PY{n}{push}\PY{p}{(}\PY{l+s+s1}{\PYZsq{}}\PY{l+s+s1}{dog}\PY{l+s+s1}{\PYZsq{}}\PY{p}{)}\PY{p}{,} \PY{n}{s}\PY{o}{.}\PY{n}{items}\PY{p}{)}
\end{Verbatim}


    \begin{Verbatim}[commandchars=\\\{\}]
None [4, 'dog']

    \end{Verbatim}

    \begin{Verbatim}[commandchars=\\\{\}]
{\color{incolor}In [{\color{incolor}6}]:} \PY{n+nb}{print}\PY{p}{(}\PY{n}{s}\PY{o}{.}\PY{n}{peek}\PY{p}{(}\PY{p}{)}\PY{p}{,} \PY{n}{s}\PY{o}{.}\PY{n}{items}\PY{p}{)}
\end{Verbatim}


    \begin{Verbatim}[commandchars=\\\{\}]
dog [4, 'dog']

    \end{Verbatim}

    \begin{Verbatim}[commandchars=\\\{\}]
{\color{incolor}In [{\color{incolor}7}]:} \PY{n+nb}{print}\PY{p}{(}\PY{n}{s}\PY{o}{.}\PY{n}{push}\PY{p}{(}\PY{k+kc}{True}\PY{p}{)}\PY{p}{,} \PY{n}{s}\PY{o}{.}\PY{n}{items}\PY{p}{)}
\end{Verbatim}


    \begin{Verbatim}[commandchars=\\\{\}]
None [4, 'dog', True]

    \end{Verbatim}

    \begin{Verbatim}[commandchars=\\\{\}]
{\color{incolor}In [{\color{incolor}8}]:} \PY{n+nb}{print}\PY{p}{(}\PY{n}{s}\PY{o}{.}\PY{n}{size}\PY{p}{(}\PY{p}{)}\PY{p}{,} \PY{n}{s}\PY{o}{.}\PY{n}{items}\PY{p}{)}
\end{Verbatim}


    \begin{Verbatim}[commandchars=\\\{\}]
3 [4, 'dog', True]

    \end{Verbatim}

    \begin{Verbatim}[commandchars=\\\{\}]
{\color{incolor}In [{\color{incolor}9}]:} \PY{n+nb}{print}\PY{p}{(}\PY{n}{s}\PY{o}{.}\PY{n}{isEmpty}\PY{p}{(}\PY{p}{)}\PY{p}{,} \PY{n}{s}\PY{o}{.}\PY{n}{items}\PY{p}{)}
\end{Verbatim}


    \begin{Verbatim}[commandchars=\\\{\}]
False [4, 'dog', True]

    \end{Verbatim}

    \begin{Verbatim}[commandchars=\\\{\}]
{\color{incolor}In [{\color{incolor}10}]:} \PY{n+nb}{print}\PY{p}{(}\PY{n}{s}\PY{o}{.}\PY{n}{push}\PY{p}{(}\PY{l+m+mf}{8.4}\PY{p}{)}\PY{p}{,} \PY{n}{s}\PY{o}{.}\PY{n}{items}\PY{p}{)}
\end{Verbatim}


    \begin{Verbatim}[commandchars=\\\{\}]
None [4, 'dog', True, 8.4]

    \end{Verbatim}

    \begin{Verbatim}[commandchars=\\\{\}]
{\color{incolor}In [{\color{incolor}11}]:} \PY{n+nb}{print}\PY{p}{(}\PY{n}{s}\PY{o}{.}\PY{n}{pop}\PY{p}{(}\PY{p}{)}\PY{p}{,} \PY{n}{s}\PY{o}{.}\PY{n}{items}\PY{p}{)}
\end{Verbatim}


    \begin{Verbatim}[commandchars=\\\{\}]
8.4 [4, 'dog', True]

    \end{Verbatim}

    \begin{Verbatim}[commandchars=\\\{\}]
{\color{incolor}In [{\color{incolor}12}]:} \PY{n+nb}{print}\PY{p}{(}\PY{n}{s}\PY{o}{.}\PY{n}{pop}\PY{p}{(}\PY{p}{)}\PY{p}{,} \PY{n}{s}\PY{o}{.}\PY{n}{items}\PY{p}{)}
\end{Verbatim}


    \begin{Verbatim}[commandchars=\\\{\}]
True [4, 'dog']

    \end{Verbatim}

    \begin{Verbatim}[commandchars=\\\{\}]
{\color{incolor}In [{\color{incolor}13}]:} \PY{n+nb}{print}\PY{p}{(}\PY{n}{s}\PY{o}{.}\PY{n}{size}\PY{p}{(}\PY{p}{)}\PY{p}{,} \PY{n}{s}\PY{o}{.}\PY{n}{items}\PY{p}{)}
\end{Verbatim}


    \begin{Verbatim}[commandchars=\\\{\}]
2 [4, 'dog']

    \end{Verbatim}

    \subsubsection{3.6 Simple Challenge : Balanced
Parentheses}\label{simple-challenge-balanced-parentheses}

    다음 예시에서 Parentheses(괄호)는 모두 쌍으로 이루어져있다.

    \begin{quote}
(()()()())\\
(((())))\\
(()((())()))
\end{quote}

    다음 예시는 쌍이 맞지 않는 괄호들이다.

    \begin{quote}
((((((())\\
()))\\
(()()(()
\end{quote}

    이제 우리는 괄호로 이루어진 string이 balanced인지 그렇지 않은지를
판별하는 알고리즘을 만들고자 한다. 괄호는 가장 마지막에 열린 괄호가 가장
빨리 닫히고, 가장 처음에 열린 괄호가 가장 마지막에 닫힌다. stack을
어떻게 이 알고리즘에 써야할지 이 특징이 알려준다.

빈 스택에서 시작해서, 열린 괄호를 만나면 stack에 push하고, 닫힌 괄호를
만나면, stack에서 pop하면 된다. 만약 string을 모두 읽었는데 stack에
item이 남아있거나, 빈 stack에서 pop을 시도한다면, 그 괄호는 balance하지
않은 괄호이다.

    \begin{Verbatim}[commandchars=\\\{\}]
{\color{incolor}In [{\color{incolor}14}]:} \PY{k}{def} \PY{n+nf}{parChecker}\PY{p}{(}\PY{n}{string}\PY{p}{)} \PY{p}{:}
             \PY{n}{s} \PY{o}{=} \PY{n}{Stack}\PY{p}{(}\PY{p}{)}
             \PY{n}{balanced} \PY{o}{=} \PY{k+kc}{True}
             \PY{n}{index} \PY{o}{=} \PY{l+m+mi}{0}
             \PY{k}{while} \PY{n}{index} \PY{o}{\PYZlt{}} \PY{n+nb}{len}\PY{p}{(}\PY{n}{string}\PY{p}{)} \PY{o+ow}{and} \PY{n}{balanced} \PY{p}{:}
                 \PY{n}{symbol} \PY{o}{=} \PY{n}{string}\PY{p}{[}\PY{n}{index}\PY{p}{]}
                 \PY{k}{if} \PY{n}{symbol} \PY{o}{==} \PY{l+s+s1}{\PYZsq{}}\PY{l+s+s1}{(}\PY{l+s+s1}{\PYZsq{}} \PY{p}{:}
                     \PY{n}{s}\PY{o}{.}\PY{n}{push}\PY{p}{(}\PY{n}{symbol}\PY{p}{)}
                 \PY{k}{else} \PY{p}{:}
                     \PY{k}{if} \PY{n}{s}\PY{o}{.}\PY{n}{isEmpty}\PY{p}{(}\PY{p}{)} \PY{p}{:}
                         \PY{n}{balanced} \PY{o}{=} \PY{k+kc}{False}
                     \PY{k}{else} \PY{p}{:}
                         \PY{n}{s}\PY{o}{.}\PY{n}{pop}\PY{p}{(}\PY{p}{)}
                 
                 \PY{n}{index} \PY{o}{=} \PY{n}{index} \PY{o}{+} \PY{l+m+mi}{1}
                 
             \PY{k}{if} \PY{n}{balanced} \PY{o+ow}{and} \PY{n}{s}\PY{o}{.}\PY{n}{isEmpty}\PY{p}{(}\PY{p}{)} \PY{p}{:}
                 \PY{k}{return} \PY{k+kc}{True}
             \PY{k}{else} \PY{p}{:}
                 \PY{k}{return} \PY{k+kc}{False}
\end{Verbatim}


    \begin{Verbatim}[commandchars=\\\{\}]
{\color{incolor}In [{\color{incolor}15}]:} \PY{n}{parChecker}\PY{p}{(}\PY{l+s+s1}{\PYZsq{}}\PY{l+s+s1}{((()))}\PY{l+s+s1}{\PYZsq{}}\PY{p}{)}
\end{Verbatim}


\begin{Verbatim}[commandchars=\\\{\}]
{\color{outcolor}Out[{\color{outcolor}15}]:} True
\end{Verbatim}
            
    \begin{Verbatim}[commandchars=\\\{\}]
{\color{incolor}In [{\color{incolor}16}]:} \PY{n}{parChecker}\PY{p}{(}\PY{l+s+s1}{\PYZsq{}}\PY{l+s+s1}{(()}\PY{l+s+s1}{\PYZsq{}}\PY{p}{)}
\end{Verbatim}


\begin{Verbatim}[commandchars=\\\{\}]
{\color{outcolor}Out[{\color{outcolor}16}]:} False
\end{Verbatim}
            
    \subsubsection{3.7 Balanced Symbols (A General
Case)}\label{balanced-symbols-a-general-case}

    괄호 판별기는 Balanced Symbols의 특별한 케이스이다. 파이썬에서도 ()와
\{\}, {[}{]}이 각 tuple, dictionary, list를 만드는 데에 사용된다. 다음의
예제들은 모두 Balanced Symbol이다.

    \begin{quote}
\{ \{ ( {[} {]} {[} {]} ) \} ( ) \}\\
{[} {[} \{ \{ ( ( ) ) \} \} {]} {]} {[} {]} {[} {]} {[} {]} ( ) \{ \}
\end{quote}

    다음의 string은 모두 not balanced string이다.

    \begin{quote}
( {[} ) {]}\\
( ( ( ) {]} ) )\\
{[} \{ ( ) {]}
\end{quote}

    \begin{Verbatim}[commandchars=\\\{\}]
{\color{incolor}In [{\color{incolor}17}]:} \PY{k}{def} \PY{n+nf}{parChecker}\PY{p}{(}\PY{n}{string}\PY{p}{)} \PY{p}{:}
             \PY{k}{def} \PY{n+nf}{matches}\PY{p}{(}\PY{n+nb}{open}\PY{p}{,} \PY{n}{close}\PY{p}{)} \PY{p}{:}
                 \PY{n}{opens} \PY{o}{=} \PY{l+s+s1}{\PYZsq{}}\PY{l+s+s1}{([}\PY{l+s+s1}{\PYZob{}}\PY{l+s+s1}{\PYZsq{}}
                 \PY{n}{closers} \PY{o}{=} \PY{l+s+s1}{\PYZsq{}}\PY{l+s+s1}{)]\PYZcb{}}\PY{l+s+s1}{\PYZsq{}}
                 \PY{k}{return} \PY{n}{opens}\PY{o}{.}\PY{n}{index}\PY{p}{(}\PY{n+nb}{open}\PY{p}{)} \PY{o}{==} \PY{n}{closers}\PY{o}{.}\PY{n}{index}\PY{p}{(}\PY{n}{close}\PY{p}{)}
             
             \PY{n}{s} \PY{o}{=} \PY{n}{Stack}\PY{p}{(}\PY{p}{)}
             \PY{n}{balanced} \PY{o}{=} \PY{k+kc}{True}
             \PY{n}{index} \PY{o}{=} \PY{l+m+mi}{0}
             \PY{k}{while} \PY{n}{index} \PY{o}{\PYZlt{}} \PY{n+nb}{len}\PY{p}{(}\PY{n}{string}\PY{p}{)} \PY{o+ow}{and} \PY{n}{balanced} \PY{p}{:}
                 \PY{n}{symbol} \PY{o}{=} \PY{n}{string}\PY{p}{[}\PY{n}{index}\PY{p}{]}
                 \PY{k}{if} \PY{n}{symbol} \PY{o+ow}{in} \PY{l+s+s1}{\PYZsq{}}\PY{l+s+s1}{([}\PY{l+s+s1}{\PYZob{}}\PY{l+s+s1}{\PYZsq{}} \PY{p}{:}
                     \PY{n}{s}\PY{o}{.}\PY{n}{push}\PY{p}{(}\PY{n}{symbol}\PY{p}{)}
                 \PY{k}{else} \PY{p}{:}
                     \PY{k}{if} \PY{n}{s}\PY{o}{.}\PY{n}{isEmpty}\PY{p}{(}\PY{p}{)} \PY{p}{:}
                         \PY{n}{balanced} \PY{o}{=} \PY{k+kc}{False}
                     \PY{k}{else} \PY{p}{:}
                         \PY{n}{top} \PY{o}{=} \PY{n}{s}\PY{o}{.}\PY{n}{pop}\PY{p}{(}\PY{p}{)}
                         \PY{k}{if} \PY{o+ow}{not} \PY{n}{matches}\PY{p}{(}\PY{n}{top}\PY{p}{,} \PY{n}{symbol}\PY{p}{)} \PY{p}{:}
                             \PY{n}{balanced} \PY{o}{=} \PY{k+kc}{False}
                 \PY{n}{index} \PY{o}{=} \PY{n}{index} \PY{o}{+} \PY{l+m+mi}{1}
                 
             \PY{k}{if} \PY{n}{balanced} \PY{o+ow}{and} \PY{n}{s}\PY{o}{.}\PY{n}{isEmpty}\PY{p}{(}\PY{p}{)} \PY{p}{:}
                 \PY{k}{return} \PY{k+kc}{True}
             \PY{k}{else} \PY{p}{:}
                 \PY{k}{return} \PY{k+kc}{False}
\end{Verbatim}


    \begin{Verbatim}[commandchars=\\\{\}]
{\color{incolor}In [{\color{incolor}18}]:} \PY{n+nb}{print}\PY{p}{(}\PY{n}{parChecker}\PY{p}{(}\PY{l+s+s1}{\PYZsq{}}\PY{l+s+s1}{\PYZob{}\PYZob{}}\PY{l+s+s1}{([][])\PYZcb{}()\PYZcb{}}\PY{l+s+s1}{\PYZsq{}}\PY{p}{)}\PY{p}{)}
\end{Verbatim}


    \begin{Verbatim}[commandchars=\\\{\}]
True

    \end{Verbatim}

    \begin{Verbatim}[commandchars=\\\{\}]
{\color{incolor}In [{\color{incolor}19}]:} \PY{n+nb}{print}\PY{p}{(}\PY{n}{parChecker}\PY{p}{(}\PY{l+s+s1}{\PYZsq{}}\PY{l+s+s1}{[}\PY{l+s+s1}{\PYZob{}}\PY{l+s+s1}{()]}\PY{l+s+s1}{\PYZsq{}}\PY{p}{)}\PY{p}{)}
\end{Verbatim}


    \begin{Verbatim}[commandchars=\\\{\}]
False

    \end{Verbatim}

    \subsubsection{3.8 Converting Decimal Numbers to Binary
Numbers}\label{converting-decimal-numbers-to-binary-numbers}

    컴퓨터를 다루다보면 분명 이진수를 접하게 될 것이다. 십진수 233은 이진수
11101001에 해당한다. 또, 각각 다음과 같이 표현할 수 있다.

\(2 \times 10^2 + 3 \times 10^1 + 3 \times 10^0\)

\(1 \times 2^7 + 1 \times 2^6 + 1 \times 2^5 + 0 \times 2^4 + 1 \times 2^3 + 0 \times 2^2 + 0 \times 2^1 + 1 \times 2^0\)

    십진수를 이진수로 바꾸는 것은, 2로 나눗셈을 하며 계산할 수 있다. 첫
나눗셈은 해당 수가 짝수인지, 홀수인지를 결정한다. 짝수라면, 첫 나눗셈의
나머지는 0일 것이고, 이진수의 첫 자리(가장 오른 쪽 수)는 0이 될 것이다.
나눗셈을 반복하며 나머지가 발생하면(홀수) 해당 자리수에 1을, 나머지가
없다면(짝수) 0을 적어내려간다.

    \begin{verbatim}
<figcaption>Figure 1. Decimal To Binary</figcaption>
\end{verbatim}

    \begin{Verbatim}[commandchars=\\\{\}]
{\color{incolor}In [{\color{incolor}20}]:} \PY{k}{def} \PY{n+nf}{divideBy2}\PY{p}{(}\PY{n}{decimal}\PY{p}{)} \PY{p}{:}
             \PY{n}{remstack} \PY{o}{=} \PY{n}{Stack}\PY{p}{(}\PY{p}{)}
         
             \PY{k}{while} \PY{n}{decimal} \PY{o}{\PYZgt{}} \PY{l+m+mi}{0} \PY{p}{:}
                 \PY{n}{rem} \PY{o}{=} \PY{n}{decimal} \PY{o}{\PYZpc{}} \PY{l+m+mi}{2}
                 \PY{n}{remstack}\PY{o}{.}\PY{n}{push}\PY{p}{(}\PY{n}{rem}\PY{p}{)}
                 \PY{n}{decimal} \PY{o}{=} \PY{n}{decimal} \PY{o}{/}\PY{o}{/} \PY{l+m+mi}{2}
         
             \PY{n}{binString} \PY{o}{=} \PY{l+s+s2}{\PYZdq{}}\PY{l+s+s2}{\PYZdq{}}
             \PY{k}{while} \PY{o+ow}{not} \PY{n}{remstack}\PY{o}{.}\PY{n}{isEmpty}\PY{p}{(}\PY{p}{)} \PY{p}{:}
                 \PY{n}{binString} \PY{o}{=} \PY{n}{binString} \PY{o}{+} \PY{n+nb}{str}\PY{p}{(}\PY{n}{remstack}\PY{o}{.}\PY{n}{pop}\PY{p}{(}\PY{p}{)}\PY{p}{)}
         
             \PY{k}{return} \PY{n}{binString}
\end{Verbatim}


    \begin{Verbatim}[commandchars=\\\{\}]
{\color{incolor}In [{\color{incolor}21}]:} \PY{n+nb}{print}\PY{p}{(}\PY{n}{divideBy2}\PY{p}{(}\PY{l+m+mi}{42}\PY{p}{)}\PY{p}{)}
\end{Verbatim}


    \begin{Verbatim}[commandchars=\\\{\}]
101010

    \end{Verbatim}

    10진수를 변경하는 것은 기본적으로 base로 나누는 것에서 시작하는 것이기
때문에 같은 로직을 이용해 16진수까지 바꿀 수 있는 알고리즘을 만들 수
있습니다. 10부터 15까지의 숫자를 A부터 F로 대응하여 표현할 수 있습니다.

    \begin{Verbatim}[commandchars=\\\{\}]
{\color{incolor}In [{\color{incolor}22}]:} \PY{k}{def} \PY{n+nf}{baseConverter}\PY{p}{(}\PY{n}{decimal}\PY{p}{,} \PY{n}{base}\PY{p}{)} \PY{p}{:}
             \PY{n}{digits} \PY{o}{=} \PY{l+s+s2}{\PYZdq{}}\PY{l+s+s2}{0123456789ABCDEF}\PY{l+s+s2}{\PYZdq{}}
         
             \PY{n}{remstack} \PY{o}{=} \PY{n}{Stack}\PY{p}{(}\PY{p}{)}
         
             \PY{k}{while} \PY{n}{decimal} \PY{o}{\PYZgt{}} \PY{l+m+mi}{0} \PY{p}{:}
                 \PY{n}{rem} \PY{o}{=} \PY{n}{decimal} \PY{o}{\PYZpc{}} \PY{n}{base}
                 \PY{n}{remstack}\PY{o}{.}\PY{n}{push}\PY{p}{(}\PY{n}{rem}\PY{p}{)}
                 \PY{n}{decimal} \PY{o}{=} \PY{n}{decimal} \PY{o}{/}\PY{o}{/} \PY{n}{base}
         
             \PY{n}{newString} \PY{o}{=} \PY{l+s+s2}{\PYZdq{}}\PY{l+s+s2}{\PYZdq{}}
             \PY{k}{while} \PY{o+ow}{not} \PY{n}{remstack}\PY{o}{.}\PY{n}{isEmpty}\PY{p}{(}\PY{p}{)} \PY{p}{:}
                 \PY{n}{newString} \PY{o}{=} \PY{n}{newString} \PY{o}{+} \PY{n}{digits}\PY{p}{[}\PY{n}{remstack}\PY{o}{.}\PY{n}{pop}\PY{p}{(}\PY{p}{)}\PY{p}{]}
         
             \PY{k}{return} \PY{n}{newString}
\end{Verbatim}


    \begin{Verbatim}[commandchars=\\\{\}]
{\color{incolor}In [{\color{incolor}23}]:} \PY{n+nb}{print}\PY{p}{(}\PY{n}{baseConverter}\PY{p}{(}\PY{l+m+mi}{25}\PY{p}{,}\PY{l+m+mi}{16}\PY{p}{)}\PY{p}{)}
\end{Verbatim}


    \begin{Verbatim}[commandchars=\\\{\}]
19

    \end{Verbatim}

    \begin{Verbatim}[commandchars=\\\{\}]
{\color{incolor}In [{\color{incolor}24}]:} \PY{n+nb}{print}\PY{p}{(}\PY{n}{baseConverter}\PY{p}{(}\PY{l+m+mi}{315}\PY{p}{,}\PY{l+m+mi}{16}\PY{p}{)}\PY{p}{)}
\end{Verbatim}


    \begin{Verbatim}[commandchars=\\\{\}]
13B

    \end{Verbatim}

    \subsubsection{3.10 What is a Queue?}\label{what-is-a-queue}

    Queue는 'rear'라고 불리는 하나의 end에서 아이템 삽입이 이루어지고,
'front'라 불리는 end에서 아이템 제거(배출)가 일어난다. Stack과는 다르게
\textbf{FIFO(First in First out)} 방식의 자료구조이다.

이를 설명할 수 있는 가장 쉬운 예제 중 하나는, 우리가 무언가를 하기
위해(물건을 계산하거나 영화를 관람하는 것) 줄을 서는 것을 생각하면 된다.
먼저 줄을 선 사람(item)이 먼저 입장(served)을 하고, 늦게 줄을 선 사람이
나중에 입장한다.

    \begin{verbatim}
<figcaption>Figure 2. Concept of Queue</figcaption>
\end{verbatim}

    컴퓨터에서도 queue 형태의 자료구조가 사용된다. 우리가 랩실에서 무언가를
프린트하려고 할 때, 프린터의 queue에 작업을 요청받은 순서대로 작업을
쌓고, 요청 순선대로 프린트 처리를 한다.

    \subsubsection{3.11 The Queue Abstract Data
Type}\label{the-queue-abstract-data-type}

    Queue abstract data type은 다음과 같은 구조와 method를 갖는다.

    \begin{itemize}
\tightlist
\item
  \texttt{Queue()} : 비어있는 새로운 queue 객체를 만든다. 파라미터가
  필요하지 않다.
\item
  \texttt{enqueue(item)} : 큐의 rear(뒤쪽 end)에 새로운 아이템을
  추가한다.
\item
  \texttt{dequeue()} : 큐의 front에 있는 item을 반환하고, queue에서
  아이템을 삭제한다.
\item
  \texttt{isEmpty()} : 큐가 비어있는지 여부를 boolean으로 반환한다.
\item
  \texttt{size()} : 큐에 있는 아이템 수를 반환한다.
\end{itemize}

    \begin{verbatim}
<figcaption>Table 1. Methods of queue</figcaption>
\end{verbatim}

    \begin{Verbatim}[commandchars=\\\{\}]
{\color{incolor}In [{\color{incolor}25}]:} \PY{k+kn}{from} \PY{n+nn}{structures}\PY{n+nn}{.}\PY{n+nn}{Queue} \PY{k}{import} \PY{n}{Queue}
\end{Verbatim}


    \begin{Verbatim}[commandchars=\\\{\}]
{\color{incolor}In [{\color{incolor}36}]:} \PY{n}{q}\PY{o}{=}\PY{n}{Queue}\PY{p}{(}\PY{p}{)}
         \PY{n}{q}\PY{o}{.}\PY{n}{enqueue}\PY{p}{(}\PY{l+m+mi}{4}\PY{p}{)}
         \PY{n}{q}\PY{o}{.}\PY{n}{enqueue}\PY{p}{(}\PY{l+s+s1}{\PYZsq{}}\PY{l+s+s1}{dog}\PY{l+s+s1}{\PYZsq{}}\PY{p}{)}
         \PY{n}{q}\PY{o}{.}\PY{n}{enqueue}\PY{p}{(}\PY{k+kc}{True}\PY{p}{)}
         \PY{n+nb}{print}\PY{p}{(}\PY{n}{q}\PY{o}{.}\PY{n}{size}\PY{p}{(}\PY{p}{)}\PY{p}{)}
\end{Verbatim}


    \begin{Verbatim}[commandchars=\\\{\}]
3

    \end{Verbatim}

    \begin{Verbatim}[commandchars=\\\{\}]
{\color{incolor}In [{\color{incolor}37}]:} \PY{n}{q}\PY{o}{.}\PY{n}{isEmpty}\PY{p}{(}\PY{p}{)}
\end{Verbatim}


\begin{Verbatim}[commandchars=\\\{\}]
{\color{outcolor}Out[{\color{outcolor}37}]:} False
\end{Verbatim}
            
    \begin{Verbatim}[commandchars=\\\{\}]
{\color{incolor}In [{\color{incolor}38}]:} \PY{n}{q}\PY{o}{.}\PY{n}{enqueue}\PY{p}{(}\PY{l+m+mf}{8.4}\PY{p}{)}
         \PY{n+nb}{print}\PY{p}{(}\PY{n}{q}\PY{o}{.}\PY{n}{items}\PY{p}{)}
\end{Verbatim}


    \begin{Verbatim}[commandchars=\\\{\}]
[8.4, True, 'dog', 4]

    \end{Verbatim}

    \begin{Verbatim}[commandchars=\\\{\}]
{\color{incolor}In [{\color{incolor}39}]:} \PY{n}{q}\PY{o}{.}\PY{n}{dequeue}\PY{p}{(}\PY{p}{)}
\end{Verbatim}


\begin{Verbatim}[commandchars=\\\{\}]
{\color{outcolor}Out[{\color{outcolor}39}]:} 4
\end{Verbatim}
            
    \begin{Verbatim}[commandchars=\\\{\}]
{\color{incolor}In [{\color{incolor}40}]:} \PY{n+nb}{print}\PY{p}{(}\PY{n}{q}\PY{o}{.}\PY{n}{items}\PY{p}{)}
\end{Verbatim}


    \begin{Verbatim}[commandchars=\\\{\}]
[8.4, True, 'dog']

    \end{Verbatim}

    \begin{Verbatim}[commandchars=\\\{\}]
{\color{incolor}In [{\color{incolor}41}]:} \PY{n}{q}\PY{o}{.}\PY{n}{dequeue}\PY{p}{(}\PY{p}{)}
\end{Verbatim}


\begin{Verbatim}[commandchars=\\\{\}]
{\color{outcolor}Out[{\color{outcolor}41}]:} 'dog'
\end{Verbatim}
            
    \begin{Verbatim}[commandchars=\\\{\}]
{\color{incolor}In [{\color{incolor}42}]:} \PY{n+nb}{print}\PY{p}{(}\PY{n}{q}\PY{o}{.}\PY{n}{items}\PY{p}{)}
\end{Verbatim}


    \begin{Verbatim}[commandchars=\\\{\}]
[8.4, True]

    \end{Verbatim}

    \subsubsection{Simulation : Hot Potato}\label{simulation-hot-potato}

    Queue가 현실에서 어떻게 사용되는지 알아보기 위해 Hot Potato라는 게임을
생각해보자. 이 게임에서, 참여자들은 가능한 한 빨리 옆 이웃에게 item을
전달해야 한다. 특정 포인트가 되면, item(potato)을 가지고 있는 참여자는
게임에서 제외된다. 한 사람만 남을 때까지 이 작업을 반복한다.

우리는 이 Hot Potato 게임을 시뮬레이션해볼 것이다. 우리의 queue는
참여자들의 이름을 input으로 받고, num이라는 parameter를 받아, num만큼
item이 전달된 뒤 potato를 가진 참여자의 이름을 큐에서 삭제할 것이다.

    \begin{verbatim}
<figcaption>Figure 3. Hot Potato Game</figcaption>
\end{verbatim}

    \begin{Verbatim}[commandchars=\\\{\}]
{\color{incolor}In [{\color{incolor}43}]:} \PY{k}{def} \PY{n+nf}{hotPotato}\PY{p}{(}\PY{n}{namelist}\PY{p}{,} \PY{n}{num}\PY{p}{)} \PY{p}{:}
             \PY{n}{simqueue} \PY{o}{=} \PY{n}{Queue}\PY{p}{(}\PY{p}{)}
             \PY{k}{for} \PY{n}{name} \PY{o+ow}{in} \PY{n}{namelist} \PY{p}{:}
                 \PY{n}{simqueue}\PY{o}{.}\PY{n}{enqueue}\PY{p}{(}\PY{n}{name}\PY{p}{)}
         
             \PY{k}{while} \PY{n}{simqueue}\PY{o}{.}\PY{n}{size}\PY{p}{(}\PY{p}{)} \PY{o}{\PYZgt{}} \PY{l+m+mi}{1} \PY{p}{:}
                 \PY{k}{for} \PY{n}{i} \PY{o+ow}{in} \PY{n+nb}{range}\PY{p}{(}\PY{n}{num}\PY{p}{)} \PY{p}{:}
                     \PY{n}{simqueue}\PY{o}{.}\PY{n}{enqueue}\PY{p}{(}\PY{n}{simqueue}\PY{o}{.}\PY{n}{dequeue}\PY{p}{(}\PY{p}{)}\PY{p}{)}
         
                 \PY{n}{eliminated} \PY{o}{=} \PY{n}{simqueue}\PY{o}{.}\PY{n}{dequeue}\PY{p}{(}\PY{p}{)}
                 \PY{n+nb}{print}\PY{p}{(}\PY{l+s+s1}{\PYZsq{}}\PY{l+s+si}{\PYZob{}\PYZcb{}}\PY{l+s+s1}{ is eliminated!}\PY{l+s+s1}{\PYZsq{}}\PY{o}{.}\PY{n}{format}\PY{p}{(}\PY{n}{eliminated}\PY{p}{)}\PY{p}{)}
         
             \PY{k}{return} \PY{n}{simqueue}\PY{o}{.}\PY{n}{dequeue}\PY{p}{(}\PY{p}{)}
\end{Verbatim}


    \begin{Verbatim}[commandchars=\\\{\}]
{\color{incolor}In [{\color{incolor}44}]:} \PY{n}{hotPotato}\PY{p}{(}\PY{p}{[}\PY{l+s+s2}{\PYZdq{}}\PY{l+s+s2}{Bill}\PY{l+s+s2}{\PYZdq{}}\PY{p}{,}\PY{l+s+s2}{\PYZdq{}}\PY{l+s+s2}{David}\PY{l+s+s2}{\PYZdq{}}\PY{p}{,}\PY{l+s+s2}{\PYZdq{}}\PY{l+s+s2}{Susan}\PY{l+s+s2}{\PYZdq{}}\PY{p}{,}\PY{l+s+s2}{\PYZdq{}}\PY{l+s+s2}{Jane}\PY{l+s+s2}{\PYZdq{}}\PY{p}{,}\PY{l+s+s2}{\PYZdq{}}\PY{l+s+s2}{Kent}\PY{l+s+s2}{\PYZdq{}}\PY{p}{,}\PY{l+s+s2}{\PYZdq{}}\PY{l+s+s2}{Brad}\PY{l+s+s2}{\PYZdq{}}\PY{p}{]}\PY{p}{,}\PY{l+m+mi}{7}\PY{p}{)}
\end{Verbatim}


    \begin{Verbatim}[commandchars=\\\{\}]
David is eliminated!
Kent is eliminated!
Jane is eliminated!
Bill is eliminated!
Brad is eliminated!

    \end{Verbatim}

\begin{Verbatim}[commandchars=\\\{\}]
{\color{outcolor}Out[{\color{outcolor}44}]:} 'Susan'
\end{Verbatim}
            
    \begin{verbatim}
<figcaption>Figure 4. Hot Potato Game with Queue</figcaption>
\end{verbatim}

    \paragraph{References}\label{references}

\begin{itemize}
\item
  Figure 1 :
  http://interactivepython.org/runestone/static/pythonds/BasicDS/ConvertingDecimalNumberstoBinaryNumbers.html
\item
  Figure 2 :
  http://interactivepython.org/runestone/static/pythonds/BasicDS/WhatIsaQueue.html
\item
  Figure 3, 4 :
  http://interactivepython.org/runestone/static/pythonds/BasicDS/SimulationHotPotato.html
\item
  Table 1 :
  http://interactivepython.org/runestone/static/pythonds/BasicDS/TheQueueAbstractDataType.html
\end{itemize}


    % Add a bibliography block to the postdoc
    
    
    
    \end{document}
